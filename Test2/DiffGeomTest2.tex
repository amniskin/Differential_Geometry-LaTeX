\documentclass[12pt]{amsart}

\usepackage{amsmath, amsthm, amssymb, amsfonts, comment, tikz}
\usepackage[margin=0.65 in]{geometry}
\renewcommand\thesubsection{\alph{subsection}}
\newcommand{\parti}[2]{\frac{\partial #1}{\partial #2}}
\newcommand{\inn}[1]{\left\langle #1\right\rangle}
\newcommand{\ka}[0]{\kappa}
\newcommand{\la}[0]{\lambda}
\newcommand{\na}{\nabla}

\begin{document}

\title{Differential Geometry Test (continued)}
\author{Aaron Niskin \\PID: 3337729}
\date{31 APR 2015}
\maketitle
\begin{enumerate}
	\item Curvature tensor: Given a connection $\na$, the associated curvature tensor, is a (0,3) tensor defined by,
	\begin{equation*}
	 	$R^\na(X,Y)Z=\na_X\na_YZ-\na_Y\na_XZ-\na{[X,Y]}Z$
	\end{equation*}
	If $T$ is a (0,p) tensor field, then $\na_XT$ is a (0,p) tensor defined by,
	\begin{equation*}
		(\na_XT)(X_1,...,X_p)=X(T(X_1,...,X_p))-T(\na_XX_1,X_2,...,X_p)-...-T(X_1,...,X_{p-1},\na_XX_p)
	\end{equation*}
	The sectional curvature is defined by
	\begin{equation*}
		K(X,Y)=\frac{g(R(X,Y)Y,X)}{g(X,X)g(Y,Y)-(g(X,Y))^2}
	\end{equation*}
	\item Prove the symmetries of the curvature tensor.
	\begin{enumerate}
		\item $R(X,Y)Z=\na_X\na_YZ-\na_Y\na_XZ-\na_{[X,Y]}Z=-(\na_Y\na_XZ-\na_X\na_YZ-\na_{[Y,X]}Z)=-R(Y,X)Z$
		\item For this, it is sufficient to prove that this property holds at every point. So pick a point $p\in M$, then every tangent vector can be expressed as $\sum\alpha^i\parti{}{x^i}$, where $\alpha^i$ are constants. Then, since $[\parti{}{x^i},\parti{}{x^j}]=0$, we have,
		\begin{aligh*}
			
		\end{aligh*}

	\end{enumerate}
	\newpage
	\item Schur's Theorem (6.5,6.6,6.7)

	\newpage
	\item The cross product makes $\mathbb{R}^3$ into a Lie Algebra. What Lie Group has a Lie Algebra isomorphic to it?
\end{enumerate}


\end{document}