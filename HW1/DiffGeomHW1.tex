\documentclass[12pt]{amsart}

\usepackage{amsmath, amsthm, amssymb, amsfonts, comment, tikz}
\renewcommand\thesubsection{\alph{subsection}}

\begin{document}

\title{Differential Geometry Homework 1}
\author{Aaron Niskin \\PID: 3337729}
\date{21 JAN 2015}
\maketitle

\section{}
\subsection{Suppose that a particle in 3-space is moving under a central force \textbf{F}. That is equivalent to the following condition: there is a fixed point \emph{A} such that the acceleration vector points in direction of \emph{A} at all time. Prove that its trajectory lies in a fixed plane.}
Let $r$ be a Frenet curve in $\mathbb{R}^3$ such that $\ddot r=\alpha\left(A-r\right)$ for some $\mathcal{C}^2$ function, $\alpha$ that is non-zero on $I$ and some fixed point $A$. We can define $c:=A-r$ so that $T,N,B$ are the same for both curves (up to negation). Furthermore, $\ddot c=-\ddot r=-\alpha c$.
\\\textit{Claim:} $B$ is constant, and hence $c$ must lie in the plane normal to $B$.
\begin{proof}
Firstly,
\begin{equation}
	-\alpha c=\ddot c=\frac{d^2c}{dt^2}=\frac{d}{dt}\left(\frac{dc}{ds}\frac{ds}{dt}\right)=\frac{d}{dt}\left(c'\dot s\right)=c''\dot s^2+c'\ddot s%=\dot s^2\kappa N+\ddot sT
\end{equation}
\begin{equation}
	T=c'
\end{equation}
\begin{equation}
	N=\frac{c''}{\kappa}=-\frac{\alpha c+\ddot sc'}{\dot s^2\kappa}
\end{equation}
\begin{equation}
	B=T\times N
\end{equation}
Then,
\begin{equation}
	T'=c''=\kappa N
\end{equation}
\begin{equation}
	-\kappa T+\tau B=N'=\left(-\frac{\alpha c+\ddot sc'}{\dot s^2\kappa}\right)'
\end{equation}
\begin{equation}
	B'=-\tau N
\end{equation}
Note: 
\begin{equation}
	-\kappa T+\tau B=\left(-\frac{\alpha c+\ddot sc'}{\dot s^2\kappa}\right)'=\frac{-(\alpha'c+c'\alpha+\ddot sc'')\dot s^2\kappa+\dot s^2\kappa'(\alpha c+\ddot sc')}{\dot s^4\kappa^2}
\end{equation}
By taking the right hand side of equation (8), simplifying and replacing every occurrence of $c$ with $-\frac{\dot s^2c''+\ddot sc'}{\alpha}=-\frac{\dot s^2\kappa N+\ddot sT}{\alpha}$ from (1), we find, 
\begin{align*}
	-\kappa T+\tau B=&\frac{-(\alpha'c+c'\alpha+\ddot sc'')\kappa+\dot\kappa'(\alpha c+\ddot sc')}{\dot s^2\kappa^2}
	\\=&\frac{-\alpha'(\dot s^2\kappa N+\ddot sT)\kappa/\alpha-\kappa\alpha T-\ddot s\kappa^2N-\kappa'\dot s^2\kappa N}{\dot s^2\kappa^2}
	\\=&\left(\frac{-\alpha'\ddot s-\alpha^2}{\dot s^2\kappa\alpha}\right)T-\left(\frac{\alpha'\dot s^2\kappa+\ddot s\kappa\alpha+\kappa'\alpha\dot s^2}{\dot s^2\kappa\alpha}\right)N
\end{align*}
is a linear combination of $T, N$. Since $T,N,B$ are orthonormal, we see that $\tau=0$.
\end{proof}
Furthermore, $\kappa^2\dot s^2\alpha=\alpha'\ddot s+\alpha^2$, and $\alpha'\dot s^2\kappa+\ddot s\kappa\alpha+\kappa'\alpha\dot s^2=0$
\begin{align*}
	0=&\alpha'\dot s^2\ddot s\kappa+\ddot s^2\kappa\alpha+\ddot s\kappa'\alpha\dot s^2
	\\=&(\kappa^2\dot s^2\alpha-\alpha^2)\dot s^2\kappa+\ddot s^2\kappa\alpha+\ddot s\kappa'\alpha\dot s^2
	\\=&(\kappa^2\dot s^2-\alpha)\dot s^2\kappa+\ddot s^2\kappa+\ddot s\kappa'\dot s^2
\end{align*}

\subsection{Fix the plane to be the $xy$-plane and write the equation for the trajectory of \emph{A} in polar coordinates.}
Let $x$ be in the direction of $T(s_0)$, $y$ in the direction of $N(s_0)$ and $z$ in the direction of $B$. Then $r(s)=(||c(s)||,\theta,0)+A$, where $\theta$ is the angle.

\subsection{If the force \textbf{F} is given by \textbf{F}=$\frac{cr}{||r||^3}$, show that the trajectory is part of an ellipse, hyperbola or parabola (second order or quadratic curve).}
If we assume that $\ddot c=F$, then $\ddot c(t)=F=\frac{c}{||r||^2}\frac{r}{||r||}$, then $A=\vec0$ 

\setcounter{equation}{0}
\section{A Frenet curve in $\mathbb{R}^3$ is called a \emph{Bertrand curve}, if there is a second curve such that the principle normal vectors to these two curves (at corresponding points) are identical, viewed as lines in space. One speaks in the case of a \emph{Bertrand pair of curves}. Show that non-planar Bertrand curves are characterized by the existence of a linear relation $a\kappa+b\tau=1$ with constants $a,b$, where $a\neq0$.} Let $c_1,c_2$ be a pair of Bertrand curves in $\mathbb{R}^3$ and let $N_1,N_2$ be their corresponding Principal Normal vectors at $t$. Then $\forall t\in I,$ $N_1=N_2$. Hence, $\forall t\in I$, $\dot N_1=\dot N_2$ (so we can drop the subscripts on $N$ and $\dot N$). Now, $\frac{\ddot c_1}{||\ddot c_1||}=N=\frac{\ddot c_2}{||\ddot c_2||}$
\begin{comment}Since,
\[\left(\begin{array}{c}
	T_1 \\
	N_1 \\
	B_1
\end{array} \right)' =\left(\begin{array}{ccc}
	0 & \kappa_1 & 0 \\
	-\kappa_1 & 0 & \tau_1 \\
	0 & -\tau_1 & 0
\end{array} \right)
\left(\begin{array}{c}
	T_1\\
	N_1\\
	B_1
\end{array}\right) \]
we know, $-\kappa_1T_1+\tau_1B_1=-\kappa_2T_2+\tau_2B_2$, and $\frac{T_1'}{\kappa_1}=\frac{T_2'}{\kappa_2}$
\end{comment}

\section{Suppose $r=r(t)$ is a regular curve satisfying $r''=r'\times H$ for a constant vector $H$ (according to one source this is the equation of an electron moving under a magnetic field force). Prove that $\tau$ and $\kappa$ are constants.}
Note: $r''(s)=$

\section{Let $A_x$ and $A_y$ be the operators corresponding to the cross product with the vectors $x$ and $y$ respectively. Show that $A_xA_y-A_yA_x=A_{x\times y}$}
This is just a computation. There's definitely a theory way to go here (involving commutators, or lie groups or whatever), but I can't see that way right now, so I'll just do this computationally.
\begin{comment}\begin{align*}
	\left(A_xA_y-A_yA_x\right)(z)=&x\times(y\times z)-y\times(x\times z)
	\\=&x\times\left((y_2z_3-y_3z_2)i-(y_1z_3-y_3z_1)j+(y_2z_3-y_3z_2)k \right)
	\\&- y\times\left((x_2z_3-x_3z_2)i-(x_1z_3-x_3z_1)j+(x_2z_3-x_3z_2)k \right)
	\\=&(x_2y_2z_3-x_2y_3z_2+x_3y_1z_3-x_3y_3z_1)i
	\\&-(x_1y_2z_3-x_1y_3z_2-x_3y_2z_3+x_3y_3z_2)j
	\\&+
\end{align*}
\end{comment}
\begin{align*}
	(A_XA_Y-A_YA_X)(Z)=&X\times (Y\times Z)-Y\times(X\times Z)
	\\=&X\times(Y\times Z)+Y\times(Z\times X)
	\\=&-Z\times(X\times Y)
	\\=&(X\times Y)\times Z=A_{X\times Y}(Z)
\end{align*}

\end{document}