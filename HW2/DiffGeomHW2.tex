\documentclass[12pt]{amsart}

\usepackage{amsmath, amsthm, amssymb, amsfonts, comment, tikz}
\usepackage[margin=0.65 in]{geometry}
\renewcommand\thesubsection{\alph{subsection}}

\begin{document}

\title{Differential Geometry Homework 2}
\author{Aaron Niskin \\PID: 3337729}
\date{21 JAN 2015}
\maketitle

\section{Determine the curvature and the torsion of the curve given by the intersection of the surfaces $x^3=3a^2y$ and $2xz=a^2$}
Firstly if $a=0$, then the resulting intersection is the $y,z$ plane and hence not a curve. So let $a\neq0$ which means $x,y,z\neq0$ so we can divide by them. Hence,
\begin{align*}
	8x^3z^3=&a^6 & &
	\\24a^2yz^3=&a^6 & x=&3a^2y
	\\24yz^3=&a^4 & x=&3a^2\frac{a^4}{24z^3}
	\\y=&\frac{a^4}{24z^3} & x=&\frac{a^6}{8z^3}
\end{align*}
So, we can let $z=t$, and we have that the curve is, $c(t)=\left(\frac{a^6}{8t^3},\frac{a^4}{24t^3},t\right)$ (so formally this would be two curves, one for the positive $t$ and one for negative, and note that the denominators throughout this problem cannot be zero because $t$ has to be on one side of zero). $\dot c=\left(\frac{-3a^6}{8t^4},\frac{a^4}{8t^4},1\right)$, and $||\dot c||=\sqrt{\frac{9a^{12}}{64t^8}+\frac{a^8}{64t^8}+1}$. Hence, $T=\frac{8t^4}{\sqrt{9a^{12}+a^8+64t^8}}\left(\frac{-3a^6}{8t^4},\frac{a^4}{8t^4},1\right)=\frac{1}{\sqrt{9a^{12}+a^8+64t^8}}\left(-3a^6,a^4,8t^4\right)$.
\\Next 
\begin{align*}
	\dot T=&\frac{1}{\sqrt{9a^{12}+a^8+64t^8}}\left(0,0,32t^3\right)+\frac{-512t^7}{2\sqrt{9a^{12}+a^8+64t^8}^3}\left(-3a^6,a^4,8t^4\right)
	\\=&\frac{2(9a^{12}+a^8+64t^8)}{2\sqrt{9a^{12}+a^8+64t^8}^3}\left(0,0,32t^3\right)-\frac{512t^7}{2\sqrt{9a^{12}+a^8+64t^8}^3}\left(-3a^6,a^4,8t^4\right)
	\\=&\frac{1}{2\sqrt{9a^{12}+a^8+64t^8}^3}\left(-3a^6,a^4,64t^3(9a^{12}+a^8+64t^8)-4096t^{11}\right)
	\\=&\frac{a^4}{2\sqrt{9a^{12}+a^8+64t^8}^3}\left(-3a^2,1,64a^4(9a^4+1)t^3\right)
\end{align*}
Since $T'=\dot Tt'=\frac{\dot T}{\dot s}$, where $s$ is the arc-length parameter, it suffices to find what $\frac{ds}{dt}$ ($\dot s$) is. Note, $s(t)=\int_{t_0}^t||\dot c(t)||dt=\int_{t_0}^t\sqrt{\frac{9a^{12}}{64t^8}+\frac{a^8}{64t^8}+1}dt=F(t)-F(t_0)$, where $\frac{dF}{dt}=||\dot c||=\sqrt{\frac{9a^{12}}{64t^8}+\frac{a^8}{64t^8}+1}$ (by the fundamental theorem of Calculus). Hence, $\frac{ds}{dt}=\frac{d}{dt}(F(t)-F(t_0))=\frac{dF}{dt}(t)=||\dot c||=\sqrt{\frac{9a^{12}}{64t^8}+\frac{a^8}{64t^8}+1}$. So, 
\begin{align*}
	T'=&\frac{a^4}{2\sqrt{9a^{12}+a^8+64t^8}^3}\left(-3a^2,1,64a^4(9a^4+1)t^3\right)/\sqrt{\frac{9a^{12}}{64t^8}+\frac{a^8}{64t^8}+1}
	\\=&\frac{8t^4a^4}{2\sqrt{9a^{12}+a^8+64t^8}^3}\left(-3a^2,1,64a^4(9a^4+1)t^3\right)/\sqrt{9a^{12}+a^8+64t^8}
	\\=&\frac{4t^4a^4}{\left(9a^{12}+a^8+64t^8\right)^2}\left(-3a^2,1,64a^4(9a^4+1)t^3\right)
\end{align*}
So,
\begin{align*}
	\kappa=&||c''||
	\\=&\left|\left|\frac{4t^4a^4}{\left(9a^{12}+a^8+64t^8\right)^2}\left(-3a^2,1,64a^4(9a^4+1)t^3\right)\right|\right|
	\\=&\frac{4t^4a^4}{\left(9a^{12}+a^8+64t^8\right)^2}\left|\left|\left(-3a^2,1,64a^4(9a^4+1)t^3\right)\right|\right| \text{ b/c the constant is $\geq$ 0 }
	\\=&\frac{4t^4a^4}{\left(9a^{12}+a^8+64t^8\right)^2}\left|\left|\left(-3a^2,1,64a^4(9a^4+1)t^3\right)\right|\right|
	\\=&\frac{4t^4a^4}{\left(9a^{12}+a^8+64t^8\right)^2}\sqrt{9a^4+1+4096a^8(9a^4+1)^2t^6}
\end{align*}
\begin{align*}
	\tau B=&N'+\kappa T
	\\=&\left(\frac{T'}{\kappa}\right)'+\kappa T
	\\=&\left(\frac{1}{\sqrt{9a^4+1+4096a^8(9a^4+1)^2t^6}}\left(-3a^2,1,64a^4(9a^4+1)t^3\right)\right)'+\kappa T
	\\=&\left(\frac{12288a^8(9a^4+1)^2t^5}{\dot s(9a^4+1+4096(9a^4+1)^2a^8t^6)^{3/2}}\right)\left(-3a^2,1,64a^4(9a^4+1)t^3\right)
	\\&\indent\indent+\frac{1}{\dot s\sqrt{9a^4+1+4096a^8(9a^4+1)^2t^6}}\left(0,0,192a^4(9a^4+1)t^2\right)
	\\&\indent\indent+\kappa\frac{1}{\sqrt{9a^{12}+a^8+64t^8}}\left(-3a^6,a^4,8t^4\right)
\end{align*}

\newpage
\section{If $c$ is a closed curve of length $L$ on the unit sphere, show that:}
\subsection{$\int_0^L\tau(s)ds=0$}
Firstly we note that if the curve is planar then $\tau=0$ and the proof is done. So let us assume that $\tau$ is not constantly 0.
\\Next, we shall prove that $\tau=\frac{J'}{1+J^2}$, where $J=\det\left[c,c',c''\right]$, which would lead us to (by the fundamental theorem of calculus)
\begin{align*}
	\int_0^L\tau ds=&\int_0^L\frac{J'}{1+J^2}ds=\tan^{-1}(J(L))-\tan^{-1}(J(0))
	\\=&0\text{ because the three vectors must be the same at the endpoints (closed curve)}
\end{align*}
Now to prove our claim:
\\Note that since the curve is on the unit sphere, the vectors $c,c',c\times c'$ form an orthonormal frame along the curve. Hence, $c''=\langle c'',c\rangle c+\langle c'',c'\rangle c'+\langle c'',c\times c'\rangle c\times c'$. But since $\langle c,c'\rangle=0$ (again, because it's on a sphere), we have that $0=\langle c,c'\rangle'=\langle c',c'\rangle+\langle c,c''\rangle\Rightarrow\langle c,c''\rangle=-\langle c',c'\rangle=-1$. Furthermore, $J=\det[c,c',c'']=\langle c'',c\times c'\rangle$. This last part can be seen because the determinant is the unique multi-linear map that sends $e_1,e_2,e_3$ (orthonormal basis) to 1. So that $c''=-c+Jc\times c'$ and $\kappa^2=||c''||^2=1+J^2$. K\"uhnel uses this to show that $\kappa^2>0$, but we can already see that $\kappa>1$ by considering the oscillating circle at that point (which has radius $1/\kappa$) that must have radius $\leq1$ if it's to be on $S^1$. Which means, $\kappa\geq1$. Either way, we have $\kappa>0$.
\\Furthermore, $T=c'$, $N=\frac{1}{\kappa}c''$, and $B=T\times N$. But $\langle T',c\rangle=\langle c'',c\rangle=-1\Rightarrow0=\langle T',c\rangle'=\langle T'',c\rangle+\langle T',c'\rangle=\langle T'',c\rangle+0\Rightarrow\langle T'',c\rangle=0$
\begin{align*}
	\tau=&\langle B,N'\rangle=-\langle B',N\rangle
	\\=&\frac{-1}{\kappa}\langle B',T'\rangle
	\\=&\frac{-1}{\kappa}\langle (T\times N)',T'\rangle
	\\=&\frac{-1}{\kappa}\langle (\frac{1}{\kappa}T\times T')',T'\rangle
	\\=&\frac{-1}{\kappa^2}\langle \frac{1}{\kappa}T'\times T'+\frac{1}{\kappa}T\times T''+\frac{-\kappa'}{\kappa^2}T\times T',T'\rangle
	\\=&\frac{-1}{\kappa^2}\langle\frac{1}{\kappa}T\times T''+\frac{-\kappa'}{\kappa}T\times T',T'\rangle
	\\=&\frac{-1}{\kappa}\left(\langle\frac{1}{\kappa}T\times T'',T'\rangle+\langle\frac{-\kappa'}{\kappa^2}T\times T',T'\rangle\right)
	\\=&\frac{-1}{\kappa^2}\langle T\times T'',T'\rangle
	\\=&\frac{-1}{\kappa^2}\langle T\times T'',-c+Jc\times T\rangle
	\\=&
\end{align*}
\subsection{$\int_0^L\frac{\tau}{\kappa}ds=0$}
Firstly we note that if $\tau$ is constantly zero then the integral is obviously 0. Next we will prove that $\frac{\tau}{\kappa}=\left(\frac{\kappa'}{\tau\kappa^2}\right)'$, so by the fundamental theorem of calculus,
\begin{align*}
	\int_0^L\frac{\tau}{\kappa}ds=&\frac{\kappa'(L)}{\tau(L)\kappa^2(L)}-\frac{\kappa'(0)}{\tau(0)\kappa^2(0)}=0
\end{align*}


\newpage
\section{Prove that for any real number $c$ there exists a closed curve $r$ of length $L$, such that $\int_0^L\tau ds=c$.}
Let $L,c\in\mathbb{R}_+$ ($\neq0$ because that case was done in question 2 and for negatives, just consider the reverse of our construction). We will define a curve, $r$ (parameterized by arc-length), in 4 parts such that $r$ has length $L$ with total torsion $c$. Part 1 will be $[0,\alpha]$ for $\alpha=L/10$, part two will be a curve on a sphere with $s\in[\alpha,\beta]$, for some $\beta\leq 3L/10$, then part three will be a plane curve for $s\in[\beta,\gamma]$ for some $\gamma<L$, and part four will be another curve on a sphere with $s\in[\gamma,L]$.
\\Part 1 ($s\in[0,\beta]$): $r(t)=(a\cos(t),a\sin(t),bt)$, for some $a\in\mathbb{R}$ (to be figured out later and assumed positive for our purposes).
\\Then, $\dot r=(-a\sin(t),a\cos(t),b)\rightarrow||\dot r||=\sqrt{a^2+b^2}$, so that $s(t)=\int_0^t\sqrt{a^2+b^2}dx=\sqrt{a^2+b^2}t$, and $r=r(s)=\left(a\cos\left(\frac{s}{\sqrt{a^2+b^2}}\right),a\sin\left(\frac{s}{\sqrt{a^2+b^2}}\right),\frac{bs}{\sqrt{a^2+b^2}}\right)$.
\\So, $T=\left(\frac{-a}{\sqrt{a^2+b^2}}\sin\left(\frac{s}{\sqrt{a^2+b^2}}\right),\frac{a}{\sqrt{a^2+b^2}}\cos\left(\frac{s}{\sqrt{a^2+b^2}}\right),\frac{b}{\sqrt{a^2+b^2}}\right)$,
\\and $r''=T'=\left(\frac{-a}{a^2+b^2}\cos\left(\frac{s}{\sqrt{a^2+b^2}}\right),\frac{-a}{a^2+b^2}\sin\left(\frac{s}{\sqrt{a^2+b^2}}\right),0\right)\Rightarrow\kappa=\frac{|a|}{a^2+b^2}=\frac{a}{a^2+b^2}$.
\\$N=\left(-\cos\left(\frac{s}{\sqrt{a^2+b^2}}\right),-\sin\left(\frac{s}{\sqrt{a^2+b^2}}\right),0\right)$.
\\Next, $B=T\times N=\left(\frac{b}{\sqrt{a^2+b^2}}\sin\left(\frac{s}{\sqrt{a^2+b^2}}\right),\frac{-b}{\sqrt{a^2+b^2}}\cos\left(\frac{s}{\sqrt{a^2+b^2}}\right),\frac{a}{\sqrt{a^2+b^2}}\right)$.
\begin{align*}
	\tau B=&N'+\kappa T
	\\=&\left(\frac{1}{\sqrt{a^2+b^2}}\sin\left(\frac{s}{\sqrt{a^2+b^2}}\right),\frac{-1}{\sqrt{a^2+b^2}}\cos\left(\frac{s}{\sqrt{a^2+b^2}}\right),0\right)+\kappa T
	\\=&\left(\frac{a^2+b^2-a^2}{\sqrt{a^2+b^2}^3}\sin\left(\frac{s}{\sqrt{a^2+b^2}}\right),\frac{-a^2-b^2+a^2}{\sqrt{a^2+b^2}^3}\cos\left(\frac{s}{\sqrt{a^2+b^2}}\right),\frac{ab}{\sqrt{a^2+b^2}^3}\right)
	\\=&\frac{b}{a^2+b^2}\left(\frac{b}{\sqrt{a^2+b^2}}\sin\left(\frac{s}{\sqrt{a^2+b^2}}\right),\frac{-b}{\sqrt{a^2+b^2}}\cos\left(\frac{s}{\sqrt{a^2+b^2}}\right),\frac{a}{\sqrt{a^2+b^2}}\right)\implies
	\\\tau=&\frac{b}{\sqrt{a^2+b^2}}
\end{align*}
Then, $\int_0^\alpha\tau ds=\frac{b\alpha}{\sqrt{a^2+b^2}}$. But we want $\alpha=2n\pi\sqrt{a^2+b^2}$, so that things will play out nicely in parts 2,3, and 4. So $\int_0^\alpha\tau ds=\frac{2n\pi\sqrt{a^2+b^2}b}{\sqrt{a^2+b^2}}=2n\pi b$. By keeping $b$ constant and letting $n\to \infty$ (hence $a\to0$) we see $\int_0^\alpha\tau ds\to\infty$. On the other hand, by setting $b=0$ (hence $2n\pi a=\alpha=L/10$) we see that $\int_0^\alpha\tau ds=0$. So, for any positive $c\in\mathbb{R}$, we can find $a,b,n$ such that $\int_0^\alpha\tau ds=c$. Next we will connect the two endpoints of this helix in a way such that the total torsion is 0.
\\Part 2 ($[\alpha,\beta]$ on the sphere): In particular, this portion will be on the oscillating sphere at the point $s=\alpha$. This will be any regular curve of length $\leq L/5$ on said sphere that ends on the great circle parallel to the plane defined by $T(\alpha)$ and the line segment connecting $r(0)$ with $r(\alpha)$ (the two ends of the helix). Let us call this endpoint of part 2 of this curve, $d$. The idea is that part 3 will be a plane curve connecting to the same point on the oscillating sphere at the bottom of the helix, thereby making that into a pseudo-closed curve on a sphere and by question 2, the total torsion 0.
\\Part 3 ($[\beta,\gamma]$) is just any plane curve that smoothly connects the point $d$ from above to the corresponding point $d'$ on the oscillating sphere at $r(0)$ in such a way that it is a continuation of part 2 on the sphere.
\\Part 4 ($[\gamma,L]$): Essentially a continuation of part two but on the oscillating sphere at $r(0)$. Note: $\int_\alpha^\beta\tau ds+\int_\gamma^L\tau ds=0$ because it is the total torsion of a closed curve on a sphere.
\\Then, $\int_0^L\tau ds=\int_0^\alpha\tau ds+\int_\alpha^\beta\tau ds+\int_\beta^\gamma\tau ds+\int_\gamma^L\tau ds=c+\left(\int_\alpha^\beta\tau ds+\int_\gamma^L\tau ds\right)=c$.



\newpage
\section{Provide a definition of convex curve in a plane and a proof of the Four Vertex Theorem (Theorem 2.33).}
``A simply closed plane curve is called \textit{convex}, if the image set of the boundary is a convex subset $C\subseteq\mathbb{R}^2$. The convexity of a subset $C$ is defined in the usual way, namely, for any two points contained in $C$, also the segment joining these two points is completely contained in $C$.''
\\In other words, a plane curve $c$ is called \textit{convex} if it is the boundary of a convex set in $\mathbb{R}^2$.
\\
\\Next up... The Four Vertex Theorem!
\\\textit{Claim}: A simply closed, regular and convex plane curve which is of class $\mathcal{C}^3$ has at least four local extremal points for its curvature $\kappa$ (such a point is referred to as a \textit{vertex}).
\begin{proof}
	Let $c$ be a simply closed, regular and convex plane curve parameterized by arc length and from the interval $[0,L]$ to the $x,y$ plane. Firstly, if $\kappa$ is constant on any interval, then every point is a vertex. So from now on, let us assume that $\kappa$ is not constant anywhere. Furthermore, since $\kappa$ is a local extremal point, $\kappa'=0$ and $\kappa$ changes sign.
	\\Note: Since $[0,L]$ is compact and $\kappa$ is continuous, by the extreme value theorem, $\kappa$ has an absolute minimum and maximum value in $[0,L]$. So two vertexes are given already.
	\\Since the curve is a closed curve, we can assume that $\kappa(0)$ is an absolute minimum and $\kappa(s_0)$ is an absolute maximum. Next, we can choose the coordinate system to be such that $c(0)$ and $c(s_0)$ are both on the $x$ axis. Since $c$ is convex, if $c(s_1)$ is also on the $x$ axis, then every point between $0$ and $s_1$, and $s_0$ and $s_1$, should be on that line too. But that contradicts the curve being regular (second derivative cannot be 0).
	\\Now, assume toward a contradiction that these are the only two points where $\kappa'$ changes sign. Then, if we write $c(s)$ as $(x(s),y(s))$, then the function $\kappa'y$ never changes sign (because of our choice of axis). By the Frenet equations we have,
	\begin{align*}
		T=&(x',y') & T'=&(x'',y'')=\kappa N & N=&(-y',x')\text{ because $T\cdot N=0$}
	\end{align*}
	By matching coordinates, we find, $x''=-y'\kappa$. Furthermore,
	\begin{align*}
		\int_0^L\kappa'yds=&\kappa y|_0^L-\int_0^L\kappa y'ds
		\\=&\int_0^Lx''ds=x'(L)-x'(0)=0
	\end{align*}
	But since $\kappa'y$ never changes sign, we know $\kappa'y$ must be constant. Since $y$ cannot be constant (in particular it cannot be constantly 0) on any interval (because the curve is regular), we find that $\kappa'$ is constantly zero. But that means $\kappa$ is constant, which is a contradiction to our criteria that $\kappa$ not be constant on any interval.
	\\Hence, there must be at least three vertexes. Since $\kappa'$ changes sign at each vertex and the curve is $L$ periodic, we know there must be an even number (if finite) of vertexes. Hence, there must be at least 4.
	\\Hence, the 4 vertex theorem.
\end{proof}

\newpage
\section{Suppose that a Frenet curve is an intersection of two regular (parameterized) surface elements. Show that if it is a line of curvature for both surfaces, then the surfaces intersect at a constant angle.}
Check page 266 of the other diff geom book.

\end{document}